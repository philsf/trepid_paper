\subsubsection*{NAME\label{NAME}\index{NAME}}


TRepid::TE

\subsubsection*{DESCRIPTION\label{DESCRIPTION}\index{DESCRIPTION}}


Transposable Elements class for TRepid

\subsubsection*{Transposition models\label{Transposition_models}\index{Transposition models}}


constant, exponential, SKR



head2 auto



Autoselects the model from transposition\_model configurable variable
defined in trepid.conf

\subsubsection*{constant model\label{constant_model}\index{constant model}}


Transposition happens with a fixed constant rate. Transposition rate
is interpreted as the quantity of new copies (c).

\subsubsection*{exponential model\label{exponential_model}\index{exponential model}}


Transposition happens with a fixed rate proportional to the ammount of
active TEs available.



Usage:
TRepid::TE::Model::exponential(c)

\subsubsection*{SKR (Struchiner, Kidwell, Ribeiro 2005)\label{SKR_Struchiner_Kidwell_Ribeiro_2005_}\index{SKR (Struchiner, Kidwell, Ribeiro 2005)}}


c(t+1) = ceil ( c(t) + c(t)*T0*U(c(t)))



The paper suggests that function U(c) take one of the following three
forms, that we implement here:



U1(c) = 2\^{}(-c/C0.5)
U2(c) = 1 - c\^{}5/(C0.5\^{}5 + c\^{}5)
U3(c) = 1 + (c - .5/C0.5)



To use the Ui function, give the i argument to the function



Synopsis: To use U1(c), with c=3 and T0=2 ::SKR(3)



Note: This function returns 
dc = c(t+1) - c(t) = ceil ( c(t)*T0*U(c(t)) )



Usage:
with

\subsubsection*{excise\label{excise}\index{excise}}


Excision happens with a fixed rate proportional to the ammount of
active TEs available.



Usage:
TRepid::TE::Model::excise(c)


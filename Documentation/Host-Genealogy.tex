\subsubsection*{NAME\label{NAME}\index{NAME}}


TRepid::Host::Genealogy

\subsubsection*{DESCRIPTION\label{DESCRIPTION}\index{DESCRIPTION}}


Genealogy tree of the population



use this class to keep track of the real genealogy of the population



use this class to reconstruct the most complete genealogy possible
from a given host. The methods will recursively track down the
parents, grand-parents, and so, until the most ancient antecessor of
the given host, and generate output in a format provided by GraphViz
(png, gif, jpeg...)

\subsubsection*{METHODS\label{METHODS}\index{METHODS}}


The following methods are available from this class. Internal methods
are usually preceded with a \_

\paragraph*{new\label{new}\index{new}}
\begin{verbatim}
 Title   : new
 Usage   :
 Function: 
 Example : 
 Returns : 
 Args    : SUBJECT, FILE, FORMAT, VERBOSE
 Requires:
\end{verbatim}
\paragraph*{\_get\_parents\label{_get_parents}\index{\ get\ parents}}
\begin{verbatim}
 Title   : _get_parents
 Usage   : 
 Function: 
 Example : 
 Returns : father and mother's names (if any)
 Args    : TRepid::Host name
 Requires:
\end{verbatim}


The \_get\_parents() method is applied recursively to discover each past
generation of the the Host's antecessors.



Note: The default name returned if there is no father or mother is 0.

\paragraph*{\_graph\_parents\label{_graph_parents}\index{\ graph\ parents}}
\begin{verbatim}
 Title   : _graph_parents
 Usage   :
\end{verbatim}
\begin{verbatim}
 Function: Create a connection between the Host (argument) and its
           parents
\end{verbatim}
\begin{verbatim}
 Example : _graph_parents("caim", "adam", "eve")
 Returns : 
 Args    : TRepid::Host' name, and parent's names
 Requires:
\end{verbatim}


The \_graph\_parents() method create the nodes and (directed) edges
between the host given as argument (by name) and its parents.

\paragraph*{find\label{find}\index{find}}
\begin{verbatim}
 Title   : find
 Usage   : 
 Function: 
 Example : 
 Returns : 
 Args    : 
 Requires: SUBJECT
\end{verbatim}


The find() method is the main Genealogy method. It constructs the most
complete genealogy possible with available population.



Currently only GraphViz.pm is supported.

\paragraph*{save\label{save}\index{save}}
\begin{verbatim}
 Title   : save
 Usage   : 
 Function: 
 Example : 
 Returns : 
 Args    : 
 Requires: FILE, FORMAT
\end{verbatim}
\paragraph*{file\label{file}\index{file}}
\begin{verbatim}
 Title   : file
 Usage   : 
 Function: 
 Example : 
 Returns : 
 Args    : 
 Requires:
\end{verbatim}
\paragraph*{format\label{format}\index{format}}
\begin{verbatim}
 Title   : format
 Usage   : 
 Function: 
 Example : 
 Returns : 
 Args    : 
 Requires:
\end{verbatim}
\paragraph*{subject\label{subject}\index{subject}}
\begin{verbatim}
 Title   : subject
 Usage   : 
 Function: 
 Example : 
 Returns : 
 Args    : 
 Requires:
\end{verbatim}
\paragraph*{verbose\label{verbose}\index{verbose}}
\begin{verbatim}
 Title   : verbose
 Usage   : 
 Function: 
 Example : 
 Returns : 
 Args    : 
 Requires:
\end{verbatim}

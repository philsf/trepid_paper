\subsection{NAME\label{NAME}\index{NAME}}


TRepid::Host

\subsection{DESCRIPTION\label{DESCRIPTION}\index{DESCRIPTION}}


Hosts class for TRepid.

\subsubsection*{OBJECT ATTRIBUTES\label{OBJECT_ATTRIBUTES}\index{OBJECT ATTRIBUTES}}


Atributes

\paragraph*{NAME\label{NAME}\index{NAME}}


Is a string, and it should be unique in the population. If the user
lets the names be auto-generated, they will be sequential integers.

\paragraph*{GENDER\label{GENDER}\index{GENDER}}


Is a single character: M for Male and F for female.

\paragraph*{FITNESS\label{FITNESS}\index{FITNESS}}


The total count of offspring this host generated so far.

\paragraph*{AGE\label{AGE}\index{AGE}}


Is an integer. Offspring are auto-generated with age 0 (as oposed to
1), and it can be increased until it reaches \$age\_max (which is set in
TRepid::Conf

\paragraph*{GM\label{GM}\index{GM}}


Is an integer. Represents the total TE count in the host's genome.

\paragraph*{STRAND1\label{STRAND1}\index{STRAND1}}


The list of insertion sites. Is a reference to an array of TRepid::TE
objects, or undef values. Strand1 is the dna strand originated from

\paragraph*{STRAND2\label{STRAND2}\index{STRAND2}}


See STRAND1

\paragraph*{FILE\label{FILE}\index{FILE}}


The original file name, if this individual existed on disk previous to
simulation start. In this case, the host's attributes are loaded from
the file (relative path).

\paragraph*{NEWFILE\label{NEWFILE}\index{NEWFILE}}


The name the file will have when it's saved. The filename will be
updated at the end of the simulation to reflect any changes in the
attributes (like age, for example).

\subsection{METHODS\label{METHODS}\index{METHODS}}


The following methods are available from this class. Internal methods
are usually preceded with a \_

\subsubsection*{Basic Methods\label{Basic_Methods}\index{Basic Methods}}
\paragraph*{new\label{new}\index{new}}
\begin{verbatim}
 Title   : new
 Usage   :
 Function: 
 Example :
 Returns :
 Args    :
 Requires: FILE || NAME, GENDER, FITNESS, AGE, PARENTS, STRAND1, STRAND2
\end{verbatim}
\paragraph*{file\label{file}\index{file}}
\begin{verbatim}
 Title   : file
 Usage   : $filename = $host->file;
           file('adam-M-50-2');
\end{verbatim}
\begin{verbatim}
 Function: Returns the filename if called without arguments
           Sets filename if called with argument
\end{verbatim}
\begin{verbatim}
 Example :
 Returns : string
 Args    : [optional] filename
 Requires:
\end{verbatim}
\paragraph*{name\label{name}\index{name}}
\begin{verbatim}
 Title   : name
\end{verbatim}
\begin{verbatim}
 Usage   : $name = name();
           name('adam');
\end{verbatim}
\begin{verbatim}
 Function: Returns the host name if called without arguments
           Sets host name if called with argument
\end{verbatim}
\begin{verbatim}
 Example :
 Returns : string
 Args    : [optional] string
 Requires:
\end{verbatim}
\paragraph*{gender\label{gender}\index{gender}}
\begin{verbatim}
 Title   : gender
\end{verbatim}
\begin{verbatim}
 Usage   : $gender = gender();
           gender('M');
\end{verbatim}
\begin{verbatim}
 Function: Returns the host gender if called without arguments
           Sets host gender if called with argument
\end{verbatim}
\begin{verbatim}
 Example :
 Returns :
 Args    :
 Requires:
\end{verbatim}


Gender can be one of 'M' or 'F'.

\paragraph*{age\label{age}\index{age}}


Title   : age



Usage   : \$age = age();
          age(2);



Function: 
Example :
Returns : integer
Args    : [optional] integer
 Requires:

\paragraph*{fitness\label{fitness}\index{fitness}}
\begin{verbatim}
 Title   : fitness
 Usage   : $fitness = fitness()
          fitness(50)
\end{verbatim}
\begin{verbatim}
 Function: 
 Example :
 Returns : integer
 Args    : [optional] integer
 Requires:
\end{verbatim}


The number of offpsring this ::Host has generated so far. It will be
updated by Population::offspring()

\paragraph*{strand1\label{strand1}\index{strand1}}
\begin{verbatim}
 Title   : strand1
 Usage   :
 Function: 
 Example :
 Returns : 
 Args    : [optional] DNA strand
 Requires:
\end{verbatim}
\paragraph*{strand2\label{strand2}\index{strand2}}
\begin{verbatim}
 Title   : strand2
 Usage   : 
 Function: 
 Example : 
 Returns : 
 Args    : [optional] DNA strand
 Requires:
\end{verbatim}
\paragraph*{heritage\label{heritage}\index{heritage}}
\begin{verbatim}
 Title   : heritage
 Usage   :
 Function: Defines the genetic heritage of the offspring
 Example :
 Returns : DNA strand
 Args    : [optional] two refs to DNA strands
 Requires: STRAND1, STRAND2
\end{verbatim}


The heritage() method randomly selects one DNA strand from each parent
and injects one strand from each parent into the offspring as his own
strands. Offspring's strand 1 comes from male parent, and strand 2
comes from female parent.



Each strand is a reference to an array of Bio::Seq objects, in no
particular order.

\paragraph*{recombine\label{recombine}\index{recombine}}
\begin{verbatim}
 Title   : recombine
 Usage   : 
 Function: Recombination of 
 Example : 
 Returns : DNA strand
 Args    : [optional] two refs to DNA strands
 Requires: STRAND1, STRAND2
\end{verbatim}


The recombine() method returns a gamete after a 'crossing over'
process. The gamete is just a STRAND.



The particular method of recombining is choosen in the trepid.conf
file, with the recombination\_model variable.

\paragraph*{recombine\_allstep\label{recombine_allstep}\index{recombine\ allstep}}
\begin{verbatim}
 Title   : recombine_allstep
 Usage   : 
 Function: Recombination of 
 Example : 
 Returns : DNA strand
 Args    : [optional] two refs to DNA strands
 Requires: STRAND1, STRAND2
\end{verbatim}


The recombine() method returns a gamete after a 'crossing over'
process. The gamete is just a STRAND.

\paragraph*{recombine\_2step\label{recombine_2step}\index{recombine\ 2step}}
\begin{verbatim}
 Title   : recombine_2step
 Usage   : 
 Function: Recombination of 
 Example : 
 Returns : DNA strand
 Args    : [optional] two refs to DNA strands
 Requires: STRAND1, STRAND2
\end{verbatim}


The recombine() method returns a gamete after a 'crossing over'
process. The gamete is just a STRAND.



Chooses two insertion sites and swap the strands' sites between the two sites.

\paragraph*{recombine\_1step\label{recombine_1step}\index{recombine\ 1step}}
\begin{verbatim}
 Title   : recombine_1step
 Usage   : 
 Function: Recombination of 
 Example : 
 Returns : DNA strand
 Args    : [optional] two refs to DNA strands
 Requires: STRAND1, STRAND2
\end{verbatim}


The recombine() method returns a gamete after a 'crossing over'
process. The gamete is just a STRAND.



Chooses one insertion sites and swap the strands' sites after that site.

\paragraph*{recombine\_none\label{recombine_none}\index{recombine\ none}}
\begin{verbatim}
 Title   : recombine_none
 Usage   : 
 Function: 
 Example : 
 Returns : DNA strand
 Args    : [optional] two refs to DNA strands
 Requires: STRAND1, STRAND2
\end{verbatim}


The recombine() method returns a gamete after a 'crossing over'
process. The gamete is just a STRAND.



This trivial recombination model does not recombine with crossover,
and just returns one of the two strands at random.

\paragraph*{TE\_excise\label{TE_excise}\index{TE\ excise}}
\begin{verbatim}
 Title   : TE_excise
 Usage   :
 Function: Excise a TE from the Host's genome
 Example :
 Returns : Boolean
 Args    : extended site
 Requires: STRAND1, STRAND2
\end{verbatim}
\paragraph*{TE\_random\label{TE_random}\index{TE\ random}}
\begin{verbatim}
 Title   : TE_random
 Usage   : 
 Function: Select a random TE
 Example : 
 Returns : TRepid::TE object
 Args    : 
 Requires: STRAND1, STRAND2
\end{verbatim}


Selects a random TE from either STRAND1 or STRAND2. If there are no
TEs, returns false.

\paragraph*{TE\_by\_extended\_site\label{TE_by_extended_site}\index{TE\ by\ extended\ site}}
\begin{verbatim}
 Title   : TE_by_extended_site
 Usage   : 
 Function: Return a TE given its extended site
 Example : 
 Returns : TRepid::TE object
 Args    : Extended site
 Requires: STRAND1, STRAND2
\end{verbatim}
\paragraph*{TE\_active\label{TE_active}\index{TE\ active}}
\begin{verbatim}
 Title   : TE_active
 Usage   : 
 Function: Select a random active TE
 Example : 
 Returns : TRepid::TE object
 Args    : 
 Requires: STRAND1, STRAND2
\end{verbatim}


Selects a random active TE from either STRAND1 or STRAND2

\paragraph*{TE\_find\label{TE_find}\index{TE\ find}}
\begin{verbatim}
 Title   : TE_find
 Usage   : $host->TE_find('transposon123')
           $host->TE_find('^gypsy.*')
 Function: Search TE by name (perl regexp).
 Example : 
 Returns : list of extended sites
 Args    : string to match TE name
 Requires: STRAND1, STRAND2
\end{verbatim}


Case sensitive. See perldoc perlre and perldoc perlretut for more
information on perl regexps.

\paragraph*{TE\_sites\_list\label{TE_sites_list}\index{TE\ sites\ list}}
\begin{verbatim}
 Title   : TE_sites_list
 Usage   :
 Function: Return list of list of extended sites
 Example :
 Returns : array
 Args    : [optional]  two refs to DNA strands
 Requires: STRAND1, STRAND2
\end{verbatim}
\paragraph*{TE\_active\_list\label{TE_active_list}\index{TE\ active\ list}}
\begin{verbatim}
 Title   : TE_active_list
 Usage   : 
 Function: Returns the list of active TEs' extended sites 
 Example : 
 Returns : array
 Args    : 
 Requires: STRAND1, STRAND2
\end{verbatim}
\paragraph*{parents\label{parents}\index{parents}}
\begin{verbatim}
 Title   : parents
 Usage   : parents()
           parents(["adam", "eve"])
 Function: Returns the filename if called without arguments
           Sets filename if called with argument
\end{verbatim}
\begin{verbatim}
 Example :
 Returns : arrayref to (father, mother) names
 Args    : [optional] arrayref to (father, mother) names
 Requires: PARENTS
\end{verbatim}
\paragraph*{newfile\label{newfile}\index{newfile}}
\begin{verbatim}
 Title   : newfile
 Usage   : $newfile = newfile();
           newfile('adam-M-50-4')
\end{verbatim}
\begin{verbatim}
 Function: 
 Example :
 Returns :
 Args    :
 Requires:
\end{verbatim}
\paragraph*{is\_wild\label{is_wild}\index{is\ wild}}
\begin{verbatim}
 Title   : is_wild
 Usage   :
 Function: 
 Example :
 Returns : Boolean
 Args    :
 Requires:
\end{verbatim}
\paragraph*{is\_gm\label{is_gm}\index{is\ gm}}
\begin{verbatim}
 Title   : is_gm
 Usage   :
\end{verbatim}
\begin{verbatim}
 Function: Returns te TE count (non-zero is GM, zero is wild)
           Sets death by excessive TE count
\end{verbatim}
\begin{verbatim}
 Example :
 Returns : integer
 Args    : [optional] integer
 Requires:
\end{verbatim}


Note: is\_gm() does only return the TE count, NOT update it.

\paragraph*{is\_dead\label{is_dead}\index{is\ dead}}
\begin{verbatim}
 Title   : is_dead
 Usage   :
 Function: Returns/sets the dead value (non-zero is dead)
 Example :
 Returns : Boolean
 Args    : [optional] Boolean
 Requires:
\end{verbatim}
\paragraph*{is\_mature\label{is_mature}\index{is\ mature}}
\begin{verbatim}
 Title   : is_mature
 Usage   :
 Function: 
 Example :
 Returns : Boolean
 Args    :
 Requires:
\end{verbatim}
\paragraph*{age\_increment\label{age_increment}\index{age\ increment}}
\begin{verbatim}
 Title   : age_increment
 Usage   :
 Function: 
 Example :
 Returns :
 Args    :
 Requires:
\end{verbatim}


Increments age. Also updates the DEAD property, i.e., sets death by age, or transposition impact if necessary.

\paragraph*{syncFS\label{syncFS}\index{syncFS}}
\begin{verbatim}
 Title   : syncFS
 Usage   : $host->syncFS
           $host->syncFS($file)
\end{verbatim}
\begin{verbatim}
 Function: 
 Example : 
 Returns : NEWFILE (or FILE)
 Args    : [optional] file name
 Requires: NEWFILE (or FILE)
\end{verbatim}


The \_newfile() update (or newfile() directly) should be done
previously to saving the object to disk. Normally, this is handled by
Population::syncFS().



If called with a parameter, it gets assigned to NEWFILE, which is the
preferred filename to be used.

\paragraph*{\_newfile\label{_newfile}\index{\ newfile}}
\begin{verbatim}
 Title   : _newfile
 Usage   : Updates the NEWFILE attribute
 Function: 
 Example : 
 Returns : string (NEWFILE)
 Args    : 
 Requires: NAME, AGE, GENDER, FITNESS, PARENTS
\end{verbatim}


Updates the NEWFILE attribute according to current attributes. For the
accessor method, see the "newfile()" method.

\paragraph*{\_write\_seqs\_fasta\label{_write_seqs_fasta}\index{\ write\ seqs\ fasta}}
\begin{verbatim}
 Title   : _write_seqs_fasta
 Usage   : 
 Function: 
 Example : 
 Returns : string (filename)
 Args    : 
 Requires: NEWFILE||FILE, STRAND1, STRAND2
\end{verbatim}
\paragraph*{show\label{show}\index{show}}
\begin{verbatim}
 Title   : show
 Usage   : 
 Function: 
 Example : 
 Returns : string
 Args    : 
 Requires: NAME, GENDER, AGE, FITNESS, GM, NEWFILE||FILE
\end{verbatim}


This method does not require the host to be alive, so should not be
used to print individual information of whole populations. For that,
use the table\_show() method below.

\paragraph*{table\_show\label{table_show}\index{table\ show}}
\begin{verbatim}
 Title   : table_show
 Usage   : 
 Function: 
 Example : 
 Returns : string
 Args    : 
 Requires: DEAD=0, NAME, GENDER, AGE, FITNESS, GM, NEWFILE||FILE
\end{verbatim}


The table\_show() method is used to print summarized information of
individual hosts. See also TRepid::Population::show().

\paragraph*{TE\_show\label{TE_show}\index{TE\ show}}
\begin{verbatim}
 Title   : TE_show
 Usage   :
 Function: 
 Example :
 Returns : string
 Args    :
 Requires:
\end{verbatim}
\paragraph*{fitness\_impact\label{fitness_impact}\index{fitness\ impact}}
\begin{verbatim}
 Title   : fitness_impact
 Usage   : $lost_offspring = $h->fitness_impact
 Function: 
 Example : 
 Returns : integer
 Args    :
 Requires:
\end{verbatim}


Returns the mean cost of fitness before mating. The fitness cost is given as the ammount of offpsring not created when this host mates due to transposition effects.



This does not take into account offspring that are born dead because of deleterious transposition.


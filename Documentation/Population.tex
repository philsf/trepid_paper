\subsection{NAME\label{NAME}\index{NAME}}


TRepid::Population

\subsection{DESCRIPTION\label{DESCRIPTION}\index{DESCRIPTION}}


Population class for TRepid

\subsubsection*{OBJECT ATTRIBUTES\label{OBJECT_ATTRIBUTES}\index{OBJECT ATTRIBUTES}}
\paragraph*{DATADIR\label{DATADIR}\index{DATADIR}}


The directory where the population is stored on disk.

\paragraph*{POP\label{POP}\index{POP}}


Hash reference of TRepid::Host indexed by Host name.

\paragraph*{DEAD\label{DEAD}\index{DEAD}}


see POP

\paragraph*{SIZE\label{SIZE}\index{SIZE}}


The size of the population.

\paragraph*{TE\label{TE}\index{TE}}


How many TEs (total) there are in the population.

\paragraph*{FITNESS\label{FITNESS}\index{FITNESS}}


The mean fitness of the population

\paragraph*{LOG\label{LOG}\index{LOG}}


Contatenation of strings (log messages), that can be saved to logfile.

\subsection{METHODS\label{METHODS}\index{METHODS}}
\subsubsection*{Methods for population management\label{Methods_for_population_management}\index{Methods for population management}}


The following methods are available from this class.



Internal methods are usually preceded with a \_

\paragraph*{new\label{new}\index{new}}
\begin{verbatim}
 Title   : new
 Usage   : $pop->new(DATADIR=>"data")
 Function: Creates a new population
 Example :
 Returns :
 Args    : 
 Requires: DATADIR or ($data_dir in TRepid::Conf)
\end{verbatim}


The new() method initializes an empty population object, ready to be
created (with init()) or loaded from disk (with collect()).

\paragraph*{collect\label{collect}\index{collect}}
\begin{verbatim}
 Title   : collect
 Usage   : $pop->collect
 Function: Loads existing population from disk into memory
 Example :
 Returns : POP
 Args    : none
 Requires: DATADIR
\end{verbatim}


The collect() method will create one TRepid::Host object per file
found in the DATADIR directory, which is assumed to be set, and
exist. Each filename will be automatically parsed by the proper method
in the Hosts class.

\paragraph*{collect\_fast\label{collect_fast}\index{collect\ fast}}
\begin{verbatim}
 Title   : collect_fast
 Usage   : $pop->collect_fast
 Function: Loads existing population from disk into memory
 Example :
 Returns : POP
 Args    : 
 Requires: DATADIR
\end{verbatim}


Same as collect() method, but only parses Wild and GM directories (not
Deceased)

\paragraph*{init\label{init}\index{init}}
\begin{verbatim}
 Title   : init
 Usage   : $pop->init
 Function: Inits a new population in the disk
 Example :
 Returns : POP
 Args    : 
 Requires: DATADIR
\end{verbatim}
\paragraph*{clear\label{clear}\index{clear}}
\begin{verbatim}
 Title   : clear
 Usage   : $pop->clear
 Function: Clears the whole population from memory
 Example :
 Returns :
 Args    : 
 Requires: POP, DEAD
\end{verbatim}


Deletes the references for all TRepid::Host objects, forcing perl to
clear them from memory.

\paragraph*{wipe\label{wipe}\index{wipe}}
\begin{verbatim}
 Title   : wipe
 Usage   : &TRepid::Population::wipe
 Function: Fast way to clean the whole population on disk
 Example :
 Returns :
 Args    : 
 Requires:
\end{verbatim}


Fast-deletes all files in the compartment sub-directories of DATADIR.



Obs: this is a function, not a method!

\paragraph*{pop\label{pop}\index{pop}}
\begin{verbatim}
 Title   : pop
 Usage   : $ref = $population->pop
 Function: Returns the live population 
 Example :
 Returns : Reference of a hash of TRepid::Host objects
 Args    :
 Requires: POP
\end{verbatim}


The hash is indexed by the host's name, i.e., (key,value)=(name,obj).

\paragraph*{dead\label{dead}\index{dead}}
\begin{verbatim}
 Title   : dead
 Usage   : @dead = $pop->dead
 Function: Returns the dead population
 Example :
 Returns : Reference of a hash of TRepid::Host objects
 Args    :
 Requires: DEAD
\end{verbatim}


The hash is indexed by the host's name, i.e., (key,value)=(name,obj).

\paragraph*{couple\label{couple}\index{couple}}
\begin{verbatim}
 Title   : couple
 Usage   : @couple = $pop->couple
 Function: Choose Male and Female, and copulate them
 Example :
 Returns :
 Args    : list of TRepid::Host names
 Requires: POP
\end{verbatim}


Choose a random mating pair in the population, suitable for input to
offpsring().

\paragraph*{offspring\label{offspring}\index{offspring}}
\begin{verbatim}
 Title   : offspring
 Usage   : $child = $pop->offspring(@couple)
 Function: Copulate Male and Female and generate one offspring
 Example : $child = $pop->offspring("adam", "eve", "caim")
 Returns : new TRepid::Host object
\end{verbatim}
\begin{verbatim}
 Args    : two string parameters (male and female parents names)
           [optional] third argument (offspring's name)
\end{verbatim}
\begin{verbatim}
 Requires: POP
\end{verbatim}


The iteration needed for multiple offsprings is not done by this method.

\paragraph*{sample\label{sample}\index{sample}}
\begin{verbatim}
 Title   : sample
 Usage   : $host = $pop->sample
 Function: Samples the population for a random host
 Example : $gm_host = $pop->sample( $pop->gm )
 Returns : string (TRepid::Host name)
 Args    : [optional] list of TRepid::Host names
 Requires: POP
\end{verbatim}
\paragraph*{syncFS\label{syncFS}\index{syncFS}}
\begin{verbatim}
 Title   : syncFS
 Usage   : $pop->syncFS
 Function: Saves changes to Hosts' files on disk
 Example :
 Returns :
 Args    :
 Requires: POP, DEAD
\end{verbatim}


The syncFS() method uses the TRepid::Host::syncFS() method to save each
host into its new filename.

\paragraph*{host\label{host}\index{host}}
\begin{verbatim}
 Title   : host
 Usage   : $host = $pop->host("adam")
 Function: Return TRepid::Host object, by name
 Example : 
 Returns : TRepid::Host object
 Args    : string (name)
 Requires: POP
\end{verbatim}


Easy method to retrieve a host by its name.

\subsubsection*{Methods that return list of names satisfying attribute choice\label{Methods_that_return_list_of_names_satisfying_attribute_choice}\index{Methods that return list of names satisfying attribute choice}}
\paragraph*{wild\label{wild}\index{wild}}
\begin{verbatim}
 Title   : wild
 Usage   : @names = $pop->wild
 Function: 
 Example :
 Returns : List of TRepid::Host names
 Args    : 
 Requires: POP
\end{verbatim}
\paragraph*{gm\label{gm}\index{gm}}
\begin{verbatim}
 Title   : gm
 Usage   : @names = $pop->gm
 Function: 
 Example : 
 Returns : List of TRepid::Host names
 Args    : 
 Requires: POP
\end{verbatim}
\paragraph*{are\_dead\label{are_dead}\index{are\ dead}}
\begin{verbatim}
 Title   : are_dead
 Usage   : @names = $pop->are_dead
 Function: 
 Example : 
 Returns : List of TRepid::Host names
 Args    : 
 Requires: DEAD
\end{verbatim}


Cemetery. See the \_dead() method to update the list, prior to returning it.

\paragraph*{not\_dead\label{not_dead}\index{not\ dead}}
\begin{verbatim}
 Title   : not_dead
 Usage   : @names = $pop->not_dead
 Function: 
 Example : 
 Returns : List of TRepid::Host names
 Args    : 
 Requires: POP
\end{verbatim}


not\_dead() returns a list of names, while size() returns an integer.

\paragraph*{males\label{males}\index{males}}
\begin{verbatim}
 Title   : males
 Usage   : @names = $pop->males
 Function: 
 Example : 
 Returns : list of names
 Args    : 
 Requires:
\end{verbatim}
\paragraph*{females\label{females}\index{females}}
\begin{verbatim}
 Title   : females
 Usage   : @names = $pop->females
 Function: 
 Example : 
 Returns : list of names
 Args    : 
 Requires:
\end{verbatim}
\paragraph*{mature\label{mature}\index{mature}}
\begin{verbatim}
 Title   : mature
 Usage   : @names = $pop->mature
 Function: Returns list of names of sexually mature TRepid::Host
 Example :
 Returns : integer
 Args    :
 Requires: POP
\end{verbatim}
\paragraph*{are\_gender\label{are_gender}\index{are\ gender}}
\begin{verbatim}
 Title   : are_gender
 Usage   : @names = $pop->are_gender
 Function: 
 Example : 
 Returns : list of strings
 Args    : [optional] list of names
 Requires: POP
\end{verbatim}
\paragraph*{TE\_find\label{TE_find}\index{TE\ find}}
\begin{verbatim}
 Title   : TE_find
 Usage   : @names = $pop->TE_find("te_family1")
 Function: 
 Example : 
 Returns : list of strings
 Args    : string to match TE name
 Requires: POP
\end{verbatim}
\subsubsection*{Methods for data collection and Statistics\label{Methods_for_data_collection_and_Statistics}\index{Methods for data collection and Statistics}}
\paragraph*{size\label{size}\index{size}}
\begin{verbatim}
 Title   : size
 Usage   : $size = $pop->size
 Function: Returns the population size
 Example :
 Returns : integer
 Args    :
 Requires: POP
\end{verbatim}


The size() method only accesses the SIZE attribute. To actually update
the population size, use the \_size() method.

\paragraph*{TE\_total\label{TE_total}\index{TE\ total}}
\begin{verbatim}
 Title   : TE_total
 Usage   : $Total_TEs $pop->TE_total
 Function: 
 Example : 
 Returns : integer
 Args    : 
 Requires:
\end{verbatim}
\paragraph*{TE\_mean\label{TE_mean}\index{TE\ mean}}
\begin{verbatim}
 Title   : TE_mean
 Usage   : 
 Function: 
 Example : 
 Returns : real
 Args    : 
 Requires:
\end{verbatim}
\paragraph*{TE\_total\_active\label{TE_total_active}\index{TE\ total\ active}}
\begin{verbatim}
 Title   : TE_total_active
 Usage   : 
 Function: 
 Example : 
 Returns : integer
 Args    : 
 Requires:
\end{verbatim}
\subsubsection*{Methods for results management\label{Methods_for_results_management}\index{Methods for results management}}
\paragraph*{show\label{show}\index{show}}
\begin{verbatim}
 Title   : show
 Usage   : print $pop->show
 Function: Returns printable string for population display.
 Example :
 Returns : Formatted printable string of population for screen display.
 Args    :
 Requires: POP
\end{verbatim}


The show() method returns a printable string of TAB separated values
format of each host currently alive in the population.



* WARNING * Beware of show() on large populations - each individual
  will take one line of screen, and with several thousands of
  individuals it may not be what you want.

\paragraph*{show\_summary\label{show_summary}\index{show\ summary}}
\begin{verbatim}
 Title   : show_summary
 Usage   : print $pop->show
 Function: Returns printable selected config parameters.
 Example : 
 Returns : Formatted printable string with population summary information
 Args    : 
 Requires: POP, DEAD
\end{verbatim}
\paragraph*{log\label{log}\index{log}}
\begin{verbatim}
 Title   : log
 Usage   : $pop->log("message")
 Function: update log queue on RAM and prints message to screen
 Example : 
 Returns : 
 Args    : string (log message)
 Requires:
\end{verbatim}


The log() method updates the log queue on memory. To save it do disk, use
the synclog() method.

\paragraph*{synclog\label{synclog}\index{synclog}}
\begin{verbatim}
 Title   : synclog
 Usage   : $pop->synclog
 Function: 
 Example : 
 Returns : 
 Args    : 
 Requires: LOG
\end{verbatim}


The synclog() method saves the log queue to disk in the file define in
the \$log\_file var.

\subsubsection*{Methods that update population attributes\label{Methods_that_update_population_attributes}\index{Methods that update population attributes}}
\paragraph*{fitness\label{fitness}\index{fitness}}
\begin{verbatim}
 Title   : fitness
 Usage   :
 Function: Returns the population fitness
 Example :
 Returns : real
 Args    : [optional] list of TRepid::Host names
 Requires: [optional] POP
\end{verbatim}


The population fitness is defined here as the arithmetic mean of the
hosts' fitnesses (average offspring number).

\paragraph*{\_TE\_total\label{_TE_total}\index{\ TE\ total}}
\begin{verbatim}
 Title   : _TE_total
 Usage   : 
 Function: 
 Example : 
 Returns : integer
 Args    : 
 Requires:
\end{verbatim}
\paragraph*{\_size\label{_size}\index{\ size}}
\begin{verbatim}
 Title   : _size
 Usage   : 
 Function: Update SIZE
 Example : 
 Returns : integer
 Args    : 
 Requires:
\end{verbatim}
\paragraph*{\_dead\label{_dead}\index{\ dead}}
\begin{verbatim}
 Title   : _dead
 Usage   : 
 Function: Updates the DEAD hash
 Example : 
 Returns : hash ref
 Args    : 
 Requires:
\end{verbatim}
\subsubsection*{Methods that update Hosts' attributes\label{Methods_that_update_Hosts_attributes}\index{Methods that update Hosts' attributes}}
\paragraph*{age\_increment\label{age_increment}\index{age\ increment}}
\begin{verbatim}
 Title   : age_increment
 Usage   : $pop->age_increment
 Function: Increments the age of each alive host
 Example :
 Returns :
 Args    :
 Requires: POP
\end{verbatim}


Increments the age of all live hosts, remove dead hosts, updates total
TE count.

\paragraph*{\_syncfilename\label{_syncfilename}\index{\ syncfilename}}
\begin{verbatim}
 Title   : 
 Usage   : 
 Function: 
 Example : 
 Returns : 
 Args    : 
 Requires:
\end{verbatim}
\subsubsection*{Functions and methods for returning random values suitable for
attributes\label{Functions_and_methods_for_returning_random_values_suitable_for_attributes}\index{Functions and methods for returning random values suitable for
attributes}}
\paragraph*{\_rand\_age\label{_rand_age}\index{\ rand\ age}}
\begin{verbatim}
 Title   : _rand_age
 Usage   : $age = TRepid::Population::_rand_age;
 Function: 
 Example : 
 Returns : integer
 Args    : none
 Requires:
\end{verbatim}


Returns a random age integer to be used as attribute to a Host. Can
be called as a function.

\paragraph*{\_rand\_gender\label{_rand_gender}\index{\ rand\ gender}}
\begin{verbatim}
 Title   : _rand_gender
 Usage   : $gender = TRepid::Population::_rand_gender;
 Function: 
 Example : 
 Returns : gender
 Args    : none
 Requires:
\end{verbatim}


Returns a random gender string to be used as attribute to a Host. Can
be called as a function.

\paragraph*{\_rand\_host\_gender\label{_rand_host_gender}\index{\ rand\ host\ gender}}
\begin{verbatim}
 Title   : _rand_host_gender
 Usage   : _rand_host_gender("M","adam","eve")
           _rand_host_gender("F")
 Function: Returns a random name satisfying the parameter gender
 Example : 
 Returns : string
 Args    : [optional] list of names
 Requires: POP
\end{verbatim}
\paragraph*{\_nextname\label{_nextname}\index{\ nextname}}
\begin{verbatim}
 Title   : _nextname
 Usage   : 
 Function: 
 Example : 
 Returns : integer
 Args    : 
 Requires: POP, DEAD
\end{verbatim}
